\chapter{Diffusion and Resistivity}
\label{ch:Five}

\section*{Problem 5-1 (69)}
\label{sec:5-1}
By the definition of the electron diffusion coefficient, we have \(D_e \equiv \dfrac{k_BT_e}{m\nu} \), each factor can be given by
\begin{align*}
	k_BT_e =& 2\text{eV},\\
	v =& \left(\dfrac{2E}{m}\right)^{1/2} = 8.39 \times 10^5 \text{m/s},\\
	n_0 =& \dfrac{N}{V} = \dfrac{P}{k_BT_e} = 3.3\times 10^{22} \text{m}^{-1}, \\
	\nu =& n_0\overline{\sigma v} = n_0\sigma v = 1.46 \times 10^9\text{s}^{-1}.
\end{align*}
These give an electron diffusion coefficient of
\begin{equation*}
	D_e = 2.4\times10^2\text{m}^2\text{s}^{-1}.
\end{equation*}

Current density is given by \(\bm{j} = \mu n e \bm{E} \), the electron mobility can be determined using the Einstein relation
\begin{equation*}
	\mu_e = \dfrac{eD_e}{k_BT_e} = 1.2\times 10^2 \text{m}^2\text{V}^{-1}\text{s}^{-1}.
\end{equation*}
The electric field strength is therefore given by
\begin{equation*}
	E = \dfrac{j}{\mu ne} = 1.04 \times 10^4 \text{V}/\text{m}
\end{equation*}


\section*{Problem 5-2 (70)}
\label{sec:5-2}
If the plasma decays by both diffusion and recombination then the density must satisfy
\begin{equation*}
	\pp{n}{t} = D\nabla^2n - \alpha n^2.
\end{equation*}
Since we have a one-dimensional, time-independent problem, with \(n = n_0\cos(\pi x/2L)\) this simplifies to
\begin{equation*}
	D\dfrac{\partial^2n}{\partial x^2} = -D\left(\dfrac{\pi}{2L}\right)^2n = -\alpha n^2.
\end{equation*}
Solving for \(n\) yields
\begin{equation*}
	n = \dfrac{D}{\alpha} \left(\dfrac{\pi}{2L}\right)^2 = 1.1\times 10^{18}\text{m}^{-3}.
\end{equation*}

\section*{Problem 5-4 (71)}
\label{sec:5-4}
The ambipolar diffusion coefficient \(D_{a\perp} \), given by
\begin{equation*}
	D_{a\perp} \equiv \dfrac{\mu_{i\perp}D_{e\perp} + \mu_{e\perp}D_{i\perp}}{\mu_{e\perp} + \mu_{i\perp}}
\end{equation*}
can be approximated by either
\begin{equation*}
	D_{a\perp} \approx D_{i\perp} + \dfrac{\mu_{i\perp}}{\mu_{e\perp}}D_{e\perp}, \quad \text{when } \mu_{e\perp} \gg \mu_{i\perp},
\end{equation*}
or
\begin{equation*}
	D_{a\perp} \approx D_{e\perp} + \dfrac{\mu_{e\perp}}{\mu_{i\perp}}D_{i\perp}, \quad \text{when } \mu_{i\perp} \gg \mu_{e\perp}.
\end{equation*}
We know that 
\begin{equation*}
	\mu_{j\perp} = \dfrac{\pm e}{m_j\nu}\dfrac{1}{1 + \omega^2_{jc}\tau^2_j}
\end{equation*}
for the \(j\)-th species. Since we know that the electron-neutral collision cross section is the same for each species and the collision frequency \(\nu \) is proportional to the thermal velocity for each species, we will say that
\begin{equation*}
	\dfrac{\mu_e}{\mu_i} = \dfrac{eM\nu_i}{em\nu_e} \approx \left(\dfrac{M}{m}\right)^{1/2} = 85.7.
\end{equation*}
Since we are interested in the perpendicular mobility coefficients, we take
\begin{equation*}
	1 + \Omega_c^2\tau_i^2 = 1.08, \quad 1 + \omega_c^2\tau_e^2 = 580,
\end{equation*}
where we have used the fact that \(\tau = \nu^{-1} \), which was determined in problem [5-1], and that \(\Omega_c\tau_i = \omega^c\tau_e\left(\dfrac{m}{M}\right)\left(\dfrac{M}{m}\right)^{1/2} \). Putting all this together, we determine that 
\begin{equation*}
	\dfrac{\mu_{e\perp}}{\mu_{i\perp}} = \dfrac{\mu_e}{\mu_i}\dfrac{1 + \Omega_c^2\tau_i^2}{1 + \omega_c^2\tau_e^2} = 0.16 \ll 1.
\end{equation*}
Therefore, we say that 
\begin{equation*}
	D_{a\perp} \approx D_{e\perp} + \dfrac{\mu_{e\perp}}{\mu_{i\perp}}D_{i\perp} = D_{e\perp} + 0.16D_{i\perp}.
\end{equation*}
We would however, like to see if \(D_{a\perp} \approx D_{e\perp} \). To do so, we will note that since \(D = \frac{k_BT}{e}\mu \) by the Einstein relation, we can say that
\begin{equation*}
	\dfrac{D_{i\perp}}{D_{e\perp}} = \dfrac{\mu_{i\perp}}{\mu_{e\perp}}\dfrac{T_i}{T_e} = 0.3, \implies D_{i\perp} = 0.3D_{e\perp}.
\end{equation*}
Given this, we know now that
\begin{equation*}
	D_{a\perp} \approx D_{e\perp}\left[1 + (0.16)(0.3) \right] = 1.05D_{e\perp} \approx D_{e\perp.}
\end{equation*}

In order to determine the confinement time \(\tau \), we note that we are given the fact that
\begin{equation*}
	n(r=a) = 0 \;\&\; J_0(z = 2.4) = 0 \implies 2.4 = \dfrac{a}{(D\tau)^{1/2}} \implies \tau = \dfrac{1}{D_{a\perp}}\dfrac{a^2}{5.76}.
\end{equation*}
Since we solved for \(D_e\) in problem [5-1] we can compute our confinement time to be
\begin{equation*}
	\tau = \dfrac{1}{(2.4\times 10^{-2})^2}\dfrac{580}{2.4\times 10^2} = 42\mu\text{s}.
\end{equation*}

\section*{Problem 5-5 (72)}
\label{sec:5-5}
For this problem, we will equate the source and particle flux terms like so, noting that we have two walls through which particles are passing
\begin{equation*}
	2\Gamma = -D\dd{n}{x} = Q.
\end{equation*}
Our density and its derivative are given by
\begin{equation*}
	n = n_0\begin{cases}
	1 - \dfrac{x}{L} \quad x > 0 \\
	\\
	1 + \dfrac{x}{L} \quad x < 0 
	\end{cases}, \quad 
	\dd{n}{x} = n_0\begin{cases}
	 - \dfrac{1}{L} \quad x > 0 \\
	\\
	\;\;\;\dfrac{1}{L} \quad x < 0 .
	\end{cases}
\end{equation*}
Combining these results gives 
\begin{equation*}
	Q = 2\Gamma = \dfrac{2Dn_0}{L} \implies n_0 = \dfrac{QL}{2D}.
\end{equation*}
This is our expression for the peak density in terms of the problem parameters.

\section*{Problem 5-6 (73)}
\label{sec:5-6}
If the main loss mechanism is recombination, we will ignore diffusion terms and note that the equation of continuity is
\begin{equation*}
	\pp{n}{t} = -\alpha n^2,
\end{equation*}
the solution of which is given by equation [5-42]
\begin{equation*}
	\dfrac{1}{n(t)} = \dfrac{1}{n_0} + \alpha t.
\end{equation*}
Since we know that after 10msec the density has decayed to half of its initial value we can say that \(n(t = 10\text{ms} = \tau) = n_0/2 \) and our density is given by
\begin{equation*}
	\dfrac{2}{n_0} = \dfrac{1}{n_0} + \alpha \tau.
\end{equation*}
Solving for \(\alpha\) we obtain
\begin{equation*}
	\alpha = \dfrac{1}{n_0 \tau} = \dfrac{1}{(10^{20}\text{m}^{-3})(10\text{ms})} = 10^{-18}\text{m}^3/\text{s}.
\end{equation*}
This is the value of our recombination coefficient in units of meters cubed per second. 

\section*{Problem 5-7 (74)}
\label{sec:5-7}
The electron-ion mean free path is equal to the product of the electron-ion mean collision time and the relative velocity between the two species; we will approximate the velocity between the species to be the electron thermal velocity. This gives
\begin{equation*}
	\lambda_{ei} = \tau_{ei}v_{the}.
\end{equation*}
Using the definition of the electron-ion collision frequency (equation [5-62]) we obtain
\begin{equation*}
	\lambda_{ei} = \dfrac{v_{the}}{\nu_{ei}} = \dfrac{v_{the}m}{ne^2\eta}.
\end{equation*}
Lastly, we will substitute the definition of the specific resistivity \(\eta \) (equation [5-71]) to obtain
\begin{equation*}
	\lambda_{ei} = \dfrac{v_{the}m}{ne^2\eta} = \dfrac{(k_BT_e/m)^{1/2}m}{ne^2}\dfrac{(4\pi\epsilon_0)^2(k_BT_e)^{3/2}}{\pi e^2 m^{1/2}\ln \Lambda} \propto T_e^{1/2}T_e^{3/2} = T_e^2.
\end{equation*}
The mean free path for electron ion collisions is therefore proportional to the electron temperature squared.

\section*{Problem 5-8 (75)}
\label{sec:5-8}
Since our current is parallel to our magnetic field, we can simply use Ohm's law with the Spitzer resistivity, which is given by
\begin{equation*}
	\eta_{\parallel} = 5.2\times 10^{-5}\dfrac{Z\ln\Lambda}{T^{3/2}}\Omega \text{m} = (5.2\times 10^{-5})\dfrac{(1)(13.7)}{(500)^{3/2}} = 6.372\times 10^{-8} \Omega \text{m}.
\end{equation*}
Using this with Ohm's law we determine that we need an electric field of strength
\begin{equation*}
	E = \eta_{\parallel}j = \eta_{\parallel} I/A = 6.372\times 10^{-8} \Omega \text{m} \dfrac{200\text{kA}}{75\text{cm}^2} = 1.70\text{V/m}
\end{equation*}
in order to drive the current in the Tokamak as described.

\section*{Problem 5-9 (76)}
\label{sec:5-9}
The classical diffusion coefficient formula for a plasma with distinct ion and electron temperatures is given by
\begin{equation*}
	D_\perp = \dfrac{\eta n \left(k_BT_e + k_BT_i\right)}{B^2}.
\end{equation*}
Since we are interested in the perpendicular diffusion coefficient, which depends on \(\eta_\perp\), we will note that this quantity can be taken to be twice the Spitzer resistivity
\begin{equation*}
	\eta_{\perp} = 2\eta_{\parallel} = 5.2\times 10^{-5}\dfrac{Z\ln\Lambda}{T^{3/2}}\Omega \text{m} =10^{-9}\Omega \text{m},
\end{equation*}
where we have assume \(Z=1\). Using this value, along with other parameters given in the problem description we can determine the classical diffusion coefficient to be
\begin{equation*}
	D_{\perp} = 1.92\times 10^{-4}\text{m}^2\text{/s}.
\end{equation*}

For part (b), we will begin with the continuity equation and Fick's law, which say respectively that
\begin{equation*}
	\dd{N}{t} = 2\pi rL\Gamma = 2\pi r L (-D_\perp\pp{n}{r}).
\end{equation*}
We can deduce from the figure showing our density profile that
\begin{equation*}
	\pp{n}{r} = \dfrac{n}{0.1}.
\end{equation*}
Putting all of this together, we obtain
\begin{equation*}
	-\dd{N}{t} = 2\pi r L D_\perp \pp{n}{r} = (6.28)(0.50)(100)(1.92\times 10^{-4})(10^{21}/0.1) = 6.03 \times 10^{20}\text{s}^{-1}.
\end{equation*}
Lastly, we will estimate the confinement time by taking 
\begin{equation*}
	\tau = -\dfrac{N}{dN/dt} = -\dfrac{n\pi r^2 L}{dN/dt} = 150\text{s}.
\end{equation*}

%\section*{Problem 5-10}
%\label{sec:5-10}
%can we assume \(\tau \approx N/dN/dt \)?
%
%\section*{Problem 5-11}
%\label{sec:5-11}
%The classical perpendicular diffusion coefficient is given by
%\begin{equation*}
%	D_\perp = n\dfrac{k_BT_e}{B_0^2}\eta_\perp,
%\end{equation*}
%while the Bohm diffusion coefficient is given by
%\begin{equation*}
%	D_B \equiv \dfrac{k_BT_e}{16eB_0}.
%\end{equation*}
%The ratio of the of the two is
%\begin{equation*}
%	\dfrac{D_B}{D_\perp} = \dfrac{n\eta_\perp}{16eB_0} = \dfrac{(10^{19})(2.0)(5.2\times 10^{-5})(10)}{(16)(100)^{3/2}(1.602\times 10^{-19})(1)}\left(1 - \dfrac{r^2}{0.01} \right) = 1.04\times 10^{-7}\left(1 - \dfrac{r^2}{0.01} \right)
%\end{equation*}
%THIS IS PROBABLY WRONG. 

\section*{Problem 5-13 (77)}
\label{sec:5-13}
Since our plasma is being heated by a current along \(\bm{B} \), the rate of Joule heating the electrons experience is given by
\begin{equation*}
	\dd{E}{t} = \eta_{\parallel}j^2 = (5.2\times 10^{-5}) \dfrac{\ln \Lambda}{T^{3/2}}j^2 = 10^{24}\text{eV}\text{m}^{-3}\text{s}^{-1}.
\end{equation*}
Since the density does not change, the derivative of the thermal energy is given by
\begin{equation*}
	\dd{E}{t} = \dfrac{3}{2}nk_B\dd{T}{t}.
\end{equation*}
Solving for the rate of increase of \(k_BT_e \) gives
\begin{equation*}
	k_B\dd{T_e}{t} = \dfrac{3}{2}n\dd{E}{t} = 0.067 \text{eV}/\mu\text{s}.
\end{equation*}

\section*{Problem 5-15 (78)}
\label{sec:5-15}
The \(\theta\)-components of the steady-state two-fluid axis-symmetric cylindrical plasma as described by the problem can be written as
\begin{equation*}
	en(\cancel{E_\theta}- v_{ir}B) - \cancel{\nabla_\theta p_i} - e^2n^2\eta(v_{i\theta} - v_{e\theta}) = 0,
\end{equation*}
\begin{equation*}
	-en(\cancel{E_\theta} - v_{er}B) - \cancel{\nabla_\theta p_e} + e^2n^2\eta(v_{i\theta} - v_{e\theta}) = 0.
\end{equation*}
The \(\theta \) dependent terms cancel by construction. This leaves us with 
\begin{equation*}
	-en v_{ir}B - e^2n^2\eta(v_{i\theta} - v_{e\theta}) = 0,
\end{equation*}
\begin{equation*}
	env_{er}B + e^2n^2\eta(v_{i\theta} - v_{e\theta}) = 0.
\end{equation*}
Taking the sum of these two equations yields
\begin{equation*}
	-v_{ir}B + v_{er}B = 0, \implies v_{ir} = v_{er}.
\end{equation*}
Knowing this, the \(r\)-components can be written as
\begin{equation*}
	en(E_r- v_{i\theta}B) - \pp{p_i}{r} - e^2n^2\eta\cancel{(v_{ir} - v_{er})} = 0,
\end{equation*}
\begin{equation*}
	-en(E_r - v_{e\theta}B) - \pp{p_e}{r} + e^2n^2\eta\cancel{(v_{ir} - v_{er})} = 0.
\end{equation*}
Dividing through by \(enB \) and isolating the velocities yields
\begin{equation*}
	v_{i\theta} = -\dfrac{E_r}{B} + \dfrac{1}{enB}\pp{p_i}{r} \equiv v_e + v_{Di},
\end{equation*}
\begin{equation*}
	v_{e\theta} = -\dfrac{E_r}{B} - \dfrac{1}{enB}\pp{p_e}{r} \equiv v_e + v_{Di}.
\end{equation*}
Where the second equality in both equations is simply the definition of the electric field and diamagnetic drift velocities fo the guiding centers. 

Lastly, the third equation of this problem can be rearranged to show that
\begin{equation*}
	v_{ir} = - \dfrac{en\eta}{B}\left(v_{i\theta} - v_{e\theta} \right).
\end{equation*}
Using our expressions for \(v_{j\theta} \) from part (b) of this problem we obtain the expression for \(v_{ir}\) given by
\begin{equation*}
	v_{ir} = - \dfrac{\eta}{B^2}\left(\pp{p_i}{r} + \pp{p_e}{r} \right),
\end{equation*}
none of which depend on \(E_r\).

\section*{Problem 5-17 (79)}
\label{sec:5-17}
The relevant single-fluid MHD equations, with the appropriate terms neglected can be linearized as
\begin{equation*}
	\rho_0\pp{\bm{v}_1}{t} = \bm{j}_1 \times \bm{B}_0, \quad \bm{E}_1 + \bm{v}_1 \times \bm{B}_0 = \eta\bm{j}_1.
\end{equation*}
The relevant Maxwell equations are given as
\begin{equation*}
	\nabla \times \bm{E}_1 = -\bm{\dot{B}_1}, \quad \nabla \times \bm{B}_1 = \mu\bm{j}_1.
\end{equation*}
As always, we will take the curl of the first equation and substitute the second, then assume transverse waves to obtain
\begin{equation*}
	k^2\bm{E}_1 = i\omega \mu_0 \bm{j}_1.
\end{equation*}
Solving for \(\bm{v}_{1} \) by taking the cross product of the second equation with \(\bm{B}_0\) yields
\begin{equation*}
	\bm{E}_1\times\bm{B}_0 + \underbrace{(\bm{v}_1 \times \bm{B}_0)\times \bm{B}_0}_{-\bm{v}_{1\perp}B_0^2} = \eta\bm{j}_1\times\bm{B}_0,
\end{equation*}
\begin{equation*}
	 \bm{v}_1 = \dfrac{\bm{E}_1\times\bm{B}_0}{B_0^2} - \dfrac{\eta\bm{j}_1\times\bm{B}_0}{B_0^2}.
\end{equation*}
Using this equation with our first single fluid MHD equation ([4-85]) we obtain
\begin{equation*}
	-i\omega\rho_0\left(\dfrac{\bm{E}_1\times\bm{B}_0}{B_0^2} - \dfrac{\eta\bm{j}_1\times\bm{B}_0}{B_0^2} \right) = \bm{j}_1 \times \bm{B}_0.
\end{equation*}
From here, we notice that in order to satisfy Maxwell's equation above we must have \(\bm{E}\) and \(\bm{j}\) in the same direction. If we take them to both be in the \(\hat{x}\)-direction then the \(y\)-component of our equation above is
\begin{equation*}
	\dfrac{E_1}{B_0} = \left(\dfrac{iB_0}{\omega\rho_0} + \dfrac{\eta}{B_0}\right)j_1.
\end{equation*}
Using this equation and our Maxwell equation (the fifth equation in this problem) we arrive at
\begin{equation*}
	k^2E_1 = \mu_0i\omega\dfrac{E_1}{B_0}\left(\dfrac{iB_0}{\omega \rho_0} + \dfrac{\eta}{B_0}\right)^{-1} = \mu_0\omega^2\left(\dfrac{B_0^2}{\rho_0} - i\omega\eta \right)^{-1}E_1.
\end{equation*}
This equation can be rearranged to provide the dispersion relation
\begin{equation*}
	\dfrac{\omega^2}{k^2}  = c^2\epsilon_0\left(\dfrac{B_0^2}{\rho_0} - i\omega\eta \right).
\end{equation*}

In order to determine an expression for Im\((k)\) when \(\omega \) is real and \(\eta \) is small we first find an expression for \(k\)
\begin{equation*}
	k^2 = \mu\omega^2\left(\dfrac{B_0^2}{\rho_0} - i\omega\eta \right)^{-1}
\end{equation*}
\begin{equation*}
	k = \omega\left(\dfrac{\mu_u\rho_0}{B_0^2}\right)^{1/2}\left(1 - \dfrac{i\omega\eta\rho_0}{B_0^2} \right)^{-1/2}.
\end{equation*}
Some algebra yields the imaginary part
\begin{equation*}
	\text{Im}(k) = \omega\dfrac{\omega\eta\rho_0}{2B_0^2}\left(\dfrac{\mu_0\rho_0}{B_0^2}\right)^{1/2} = \dfrac{\omega^2\eta}{2}\dfrac{1}{v^3_A}
\end{equation*}
For small \(\eta\), we have Re\((k)v_A \approx \omega\), this implies that
\begin{equation*}
	\text{Im}(k) \approx \dfrac{\text{Re}(k)^2\eta}{2v_A}.
\end{equation*}






