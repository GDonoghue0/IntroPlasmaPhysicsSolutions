\chapter{Nonlinear Effects}
\label{ch:Eight}

\section*{Problem 8-1 (90)}
\label{sec:8-1}
The relationship between the probe collecting the current at saturation and the plasma density is given by equation [8-19]
\begin{equation*}
	I_B \simeq \frac{1}{2}n_0eA\sqrt{\dfrac{k_BT_e}{M}}.
\end{equation*}
Rearranging the for the sheath plasma density, we obtain
\begin{equation*}
	n_0 = \dfrac{2I}{eA} \sqrt{\dfrac{M}{k_BT}}.
\end{equation*}
Noting that both ends of the probe collect ions, we will take \(A = 2 \times 2\text{mm}\times 2\text{mm} = 8 \times 10^{-6}\text{m}^2\), and obtain
\begin{equation*}
	n_0 = \dfrac{200\mu\text{A}}{1.602\times 10^{-19}\text{C}8 \times 10^{-6}\text{m}^2} \sqrt{\dfrac{6.6335\times 10^{-26}\text{kg}}{2\text{eV}}} = 7.10\times 10^{16}\text{m}^{-3}.
\end{equation*}
We therefore say that the approximate plasma density as determined by the probe is
\begin{equation*}
	n_0 = 7.10\times 10^{16}\text{m}^{-3}.
\end{equation*}

%\section*{Problem 8-2 (91)}
%\label{sec:8-2}