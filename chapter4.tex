\chapter{Waves in Plasmas}
\label{ch:Four}

\section*{Problem 4-1 (32)}
\label{sec:4-1}
The phase of \(\phi_1 \) relative to \(n_1\) is given by
\begin{equation*}
\delta = \text{atan}\left[\dfrac{\Im(\phi_1)}{\Re(\phi_1)} \right].
\end{equation*}
We can rearrange the given expression to solve for \(\phi_1 \) as
\begin{equation*}
	\phi_1 = \dfrac{k_BT_e}{e}\dfrac{n_1}{n_0}\dfrac{\omega + ia}{\omega_* + ia} \dfrac{\omega_* - ia}{\omega_*-ia}
\end{equation*}
\begin{equation*}
	\phi_1 = \dfrac{k_BT_e}{e}\dfrac{n_1}{n_0}\dfrac{\omega\omega_* + a^2 + ia(\omega_* - \omega)}{\omega_*^2 + a^2}
\end{equation*}
Since we are assuming that \(n_1\) is real and all other symbols (except \(i\)) are positive constants we can say that
\begin{equation*}
	\delta = \text{atan}\left[\dfrac{\omega\omega_* + a^2}{a(\omega_*-\omega)} \right].
\end{equation*}
We can say that if \(\omega < \omega_* \) then we have \(\delta > 0 \). If we assume \(n_1\) to have a phase of 0 then \(\phi_1\) lags \(n_1\) in time and space.


\section*{Problem 4-2 (33)}
\label{sec:4-2}
In order to include ion motions in our plasma frequency calculation, we introduce their equations of motion
\begin{equation*}
	Mn_i\left[\pp{v_i}{t} + (v_i \cdot \nabla)v_i \right] = en_iE,
\end{equation*}
\begin{equation*}
	\pp{n_i}{t} + \nabla \cdot (n_iv_i)  = 0.
\end{equation*}
Having a uniform neutral plasma at rest implies
\begin{equation*}
	Mn_i\left[\pp{v_{i1}}{t} + (v_{i1} \cdot \nabla)v_{i1} \right] = en_{i1}E,
\end{equation*}
\begin{equation*}
	\pp{n_{i1}}{t} + n_{i0}\nabla \cdot v_{i1}  = 0.
\end{equation*}
If we assume the above quantities oscillate sinusoidally, we can take \(\pp{}{t} = -i\omega\) and \(\nabla = ik\hat{x} \) and then update equations [4-21]-[4-23] to obtain
\begin{align*}
	ikE_1 =& \dfrac{1}{\epsilon_0}e(n_{i1} - n_{e1})\\
	-i\omega m v_{e1} =& -eE_1 \\
	-i\omega M v_{i1} =& eE_1 \\
	-i\omega n_{e1} =& -ikn_0 v_{e1}\\
	-i\omega n_{i1} =& -ikn_0 v_{i1}.
\end{align*}
Using these equations we can solve for \(n_{e1} \) and \(n_{i1}\)
\begin{equation*}
n_{e1} = \dfrac{k}{\omega}n_0\left(\dfrac{-ie}{m\omega} \right)E_1, \quad n_{i1} = \dfrac{k}{\omega}n_0\left(\dfrac{ie}{m\omega} \right)E_1.
\end{equation*}
Substituting the above into Poisson's equation yields
\begin{equation*}
	ikE_1 = \dfrac{1}{\epsilon_0}\dfrac{k}{\omega} n_0\dfrac{ie}{\omega}\left(\dfrac{1}{m} + \dfrac{1}{M}\right) = \dfrac{ikE_1}{\omega^2}\left(\Omega_p^2 + \omega_p^2 \right).
\end{equation*}
We see that since \(M \gg m\) our assumption of fixed ions is justifiable.

%\section*{Problem 4-3}
%\label{sec:4-3}
%Requires drawing, skip

\section*{Problem 4-4 (34)}
\label{sec:4-4}
By manipulating our linearized equations of motion [4-21]--[4-23], we can obtain the expression
\begin{equation*}
	ikE_1 = \dfrac{ike^2n_0}{\epsilon_0m\omega^2}E_1,
\end{equation*}
which we can rearrange to obtain an expression in the form of \(\nabla \cdot \left(\epsilon E \right) = 0 \) as 
\begin{equation*}
	0 = ik\left(1 - \dfrac{e^2n_0}{\epsilon_0 m \omega^2} \right)E_1.
\end{equation*}
We can see here that an expression for high-frequency oscillations applicable to high-frequency is \(\epsilon = \left(1 - \dfrac{\omega_p^2}{\omega^2} \right) \)

\section*{Problem 4-5 (35)}
\label{sec:4-5}
We know that for electron plasma waves with thermal motions included, we have
\begin{equation*}
	\omega^2 = \omega_p^2 + \dfrac{3}{2}k^2v_{th}^2 \implies (2\pi f)^2 = \dfrac{n_0 e^2}{\epsilon_0 m} + 3k^2\dfrac{k_BT_e}{m}.
\end{equation*}
Rearranging for \(k\) gives
\begin{equation*}
	k^2 = \dfrac{\left((2\pi f)^2 - \dfrac{n_0e^2}{\epsilon_0m} \right)}{3k_BT_e/m} = 3.05 \times 10^5.
\end{equation*}
Finally we may say that our wavelength is \(\lambda = \dfrac{2\pi}{k} = 1.14 \)cm.

\section*{Problem 4-6 (36)}
\label{sec:4-6}
Our updated electron equation of motion becomes
\begin{equation*}
	mn_e\left[\pp{v_e}{t} + (v_e\cdot\nabla)v_e \right] = -en_eE - mn_e\nu v_e,
\end{equation*}
this corresponds to an updated equation [4-21] in the form of
\begin{equation*}
	-im\omega v_1 = -eE - m\nu v_1.
\end{equation*}
A useful re-expression of the above equation is
\begin{equation*}
	v_1 \left(1 + \dfrac{i\nu}{\omega} \right) = \dfrac{ieE_1}{m\omega}.
\end{equation*}
Given similarities with other problems, we will leave the rearrangement for \(\omega \) and simply state our dispersion relation as
\begin{equation*}
	\omega^2 + i\nu\omega = \omega_p^2.
\end{equation*}

If we take \(\omega = x + iy \) then our dispersion relation becomes
\begin{equation*}
	x^2 -y^2 + 2ixy + i\nu x - \nu y = \omega_p^2,
\end{equation*}
and we have \(Im(\omega) = -\nu/2\). Since \(\nu > 0 \) and \(E \propto \exp(-(ix-\nu/2)t) \), we say the wave is damped in time. 

\section*{Problem 4-7 (37)}
\label{sec:4-7}
The equations of motion for this problem ([4-55, 4-56]) tell us that for \(\bm{E} = E\hat{x}, \bm{k} = k\hat{x}, \bm{B} = B\hat{z} \), the motion is described according to
\begin{equation*}
	-i\omega m v_x = -eE - ev_yB, \quad i\omega m v_y = ev_xB.
\end{equation*}
Taking the second equation we obtain an expression for \(v_x/v_y\) in terms of \(\omega/\omega_c \)
\begin{equation*}
	\dfrac{v_x}{v_y} = -i\dfrac{\omega m}{eB} = -i\dfrac{\omega}{\omega_c}.
\end{equation*}
From here, we note that by the definition of the upper hybrid frequency \(\omega_h = \omega > \omega_c \), and therefore \(v_x > v_y \). Since \(\bm{k} = k\hat{x}\) is in the \(x\)-direction, we can say that the orbit is elongated along the direction of \(\bm{k}\)

%\section*{Problem 4-8}
%\label{sec:4-8}
%Looks hard, skip


\section*{Problem 4-9 (38)}
\label{sec:4-9}
Complete blackout will occur when the wavelength becomes so long that the wavenumber \(k = 0\) and the plasma frequency is equal to the frequency of the incident microwave. The critical density at which this occurs is given by
\begin{equation*}
	n_c = \dfrac{m\epsilon_0\omega^2}{e^2} = 1.12 \times 10^{15}\text{m}^{-3}.
\end{equation*}

\section*{Problem 4-10 (39)}
\label{sec:4-10}
Starting from the relevant equations for electromagnetic waves in a vacuum we have
\begin{equation*}
	\nabla \times \bm{E} = -\dot{\bm{B}}, \quad \nabla \times \bm{B} = \mu_0 \bm{j} + \dfrac{\dot{\bm{E}}}{c^2}.
\end{equation*}
Taking the time derivative of the second equation follow by substitution of the first gives
\begin{equation*}
	-\nabla \times \nabla \times \bm{E} = -\left(\mu_0\dot{\bm{j}} + \dfrac{\ddot{\bm{E}}}{c^2} \right)
\end{equation*}
If we assume plane waves, we have \(\nabla = \bm{k} \) and \(\ddot{(\dots)} = \omega^2 \), we can re-express the above equation as
\begin{equation*}
	-\bm{k} \times \bm{k} \times \bm{E} = -\left(\mu_0n_0e(\dot{\bm{v}}_+ - \dot{\bm{v}}_-) + \dfrac{\omega^2}{c^2}\bm{E} \right) = k^2\bm{E} - \bm{k}\cancel{(\bm{k}\cdot\bm{E})},
\end{equation*}
where in the second equation we have used the fact that transverse waves imply \(\bm{k}\cdot\bm{E} = 0 \). In order to obtain expressions for the accelerations of the electrons and positrons \(\dot{\bm{v}}_+\), and \( \dot{\bm{v}}_- \), respectively, we invoke the linearized electron equation of motion ([4-83]), and say that
\begin{equation*}
	\bm{v_\pm} = \dfrac{e\bm{E}}{im\omega} \implies \dot{\bm{v}}_\pm = \dfrac{e}{m}\bm{E} 
\end{equation*}
Combining our equations, we obtain
\begin{equation*}
	\omega^2 - c^2k^2 = \dfrac{2n_0e^2}{\epsilon_0m} = 2\omega_p^2.
\end{equation*}
We therefore say that the dispersion relation for high-frequency electromagnetic waves in our primordial universal plasma is
\begin{equation*}
	\omega^2 = 2\omega^2_p + c^2k^2.
\end{equation*} 

With regard to ions, we note that by assuming our electrons and positrons follow the Boltzmann relation, we have
\begin{equation*}
	n_\pm = n_0\exp\left(\dfrac{\mp e\phi}{k_BT_\pm} \right).
\end{equation*}
We will assume \(T_+ = T_- \equiv T_e, \), and state that our linearized density perturbation is given by
\begin{equation*}
	n_{1\pm} = \mp n_0 e  \phi /k_BT_e.
\end{equation*}
Again assuming plane waves, we can invoke our linearized equations of motion ([4-21]--[4-23]) for our proton and anti-proton motions to obtain (after some a few straightforward rearrangements)
\begin{align*}
	N_{1\pm} =& n_o\dfrac{k}{\omega}V_{\pm}\\
	V_{1\pm} =& \pm \dfrac{e{k}\phi}{m\omega} \\
	N_{1\pm} =& \pm \dfrac{k^2}{\omega^2}\dfrac{n_0e\phi}{M}.
\end{align*}
This expression can be used with Poisson's equation
\begin{equation*}
	\nabla \cdot \epsilon_0 \bm{E}_1 = \left(N_+ - N_- + n_+ - n_- \right),
\end{equation*}
to obtain
\begin{equation*}
	\nabla \cdot \bm{E} = k^2\phi = \dfrac{e}{\epsilon_0}\left[\left(2\dfrac{k^2}{\omega^2}\right)\dfrac{n_0e\phi}{M} - 2n_0\dfrac{e\phi}{k_BT_e} \right] = \dfrac{2n_0e^2k^2}{\epsilon_0\omega^2M}\phi - \dfrac{2n_0e^2}{k_BT_e}\phi.
\end{equation*}
Recognizing that \(\lambda_D^2 \equiv \dfrac{kT_e\epsilon_0}{n_0e^2} \), and \(\Omega_p^2 = \dfrac{n_0e^2}{\epsilon_0M}\), we obtain the following relation
\begin{equation*}
	k^2\phi = 2\phi\left(\Omega_p^2\dfrac{k^2}{\omega^2} - \dfrac{1}{\lambda_D^2} \right).
\end{equation*}
Rearranging the above we obtain
\begin{equation*}
	\dfrac{\omega^2}{k^2} = \dfrac{k_BT/M}{1 + k^2\lambda^2/2}
\end{equation*}
Which is the dispersion relation for the for ion waves in our equally matter and anti-matter universe.

\section*{Problem 4-11 (40)}
\label{sec:4-11}
We first note the definition of the index of refraction and the relevant plasma dielectric constant (obtained in problem 4-4), to be respectively
\begin{equation*}
	\eta \equiv \dfrac{c}{v_\phi} = c\dfrac{k}{\omega}, \quad \epsilon = 1 - \dfrac{\omega_p^2}{\omega^2}.
\end{equation*}
Starting from our dispersion relation, we can show that the index of refraction is equal to the square root of our plasma dielectric constant as follows
\begin{equation*}
	\omega^2 = \omega_p^2  + c^2k^2 \implies 1-\dfrac{\omega^2_p}{\omega^2} = \dfrac{c^2k^2}{\omega^2} \implies \epsilon = \eta^2.
\end{equation*}

\section*{Problem 4-12 (41)}
\label{sec:4-12}
Our equation for our critical density is derived from the dispersion relation
\begin{equation*}
	\omega^2 = \omega_p^2 + c^2k^2,
\end{equation*}
which assumes that the current density comes entirely from electron motion. Since we have replaced some of our electrons with chlorine ions which have too much inertia to respond to microwave signals we must update our current density equation to be
\begin{equation*}
	\bm{j}_1 = -n_0e\bm{v}_{e1}(1-\kappa),
\end{equation*}
where \(\kappa \) is the fraction of replaced electrons. This updates equation [4-84] to become 
\begin{equation*}
	\left(\omega^2 - c^2k^2 \right)\bm{E}_1 = \dfrac{n_0e^2}{\epsilon_0m}\left(1-\kappa\right)\bm{E}_1,
\end{equation*}
yielding the dispersion relation
\begin{equation*}
	\omega^2 = (1-\kappa)\omega^2_p + c^2k^2.
\end{equation*}
We know that cutoff will occur when \(k=0\) and this gives \(\omega^2_\text{cutoff} = (1-\kappa)\omega^2_p = (1-\kappa)\frac{n_0e^2}{\epsilon_0m} \). Using the fact that we seek to cutoff a \(3\)cm wave we know that \(f = \frac{c}{\lambda} = 10^{10} \)Hz. This yields a critical density of
\begin{equation*}
	n_c = 3.1\times 10^{18}\text{m}^{-3}. %\dfrac{m\epsilon_0(1-\kappa)\omega_p}{e^2}
\end{equation*}

%\section*{Problem 4-13 }
%\label{sec:4-13}
%Maybe come back to this one

\section*{Problem 4-14 (42)}
\label{sec:4-14}
The wave equation for the extraordinary wave can be expressed as in equation [4-101] to yield
\begin{equation*}
	(\omega^2 - \omega^2_h)E_x = i\dfrac{\omega^2_p\omega_c}{\omega}E_y = 0.
\end{equation*}
Resonance implies the wave frequency \(\omega \) being equal to the upper hybrid frequency \(\omega_h \), taking this to be the case, our above equation yields \(E_y = 0 \). We therefore say that since \(\bm{E} = E\hat{x}\) and \(\bm{k} = k\hat{x} \), we have \(\bm{E}\parallel\bm{k} \) and the extraordinary wave is therefore purely electrostatic at resonance.


\section*{Problem 4-15 (43)}
\label{sec:4-15}
We first hope to show that \(\omega_h < \omega_R \); we note that by setting \(k=0\) in our dispersion relation we obtain the cutoff frequency equation [4-107] which says that
\begin{equation*}
	\omega_R^2 = \omega_R\omega_c + \omega_p^2.
\end{equation*}
We furthermore note that from the definition of the right-hand cutoff frequency we have
\begin{equation*}
	\omega_R \equiv \frac{1}{2}\left[\omega_c + (\omega_c^2 + 4\omega_p^2)^{1/2} \right] > \omega_c.
\end{equation*}
This implies that 
\begin{equation*}
	\omega^2_R = \omega_R\omega_c + \omega_p^2 > \omega_c^2 + \omega^2_p \equiv \omega_h^2,
\end{equation*}
which is our rightmost inequality. Next we simply note from the definition of the upper hybrid frequency that \(\omega^2_h \equiv \omega^2_c + \omega^2_p \implies \omega_p < \omega_h \). Lastly, we state the definition of the left-hand cutoff:
\begin{equation*}
	\omega_L \equiv \frac{1}{2}\left[\omega_c - (\omega_c^2 + 4\omega_p^2)^{1/2} \right],
\end{equation*}
and since inserting a positive constant to the right side of this equation will yield an inequality we say that
\begin{equation*}
\omega_L < \frac{1}{2}\left[\omega_c - (\omega_c^2 +4\omega_c\omega_p 4\omega_p^2)^{1/2} \right] = \frac{1}{2}\left[\omega_c - (\omega_c + 2\omega_p) \right] = \omega_p.
\end{equation*}
Putting all of this together we have \(\omega_L < \omega_p < \omega_h < \omega_R \), which is how the critical points are ordered in figure 4-36.

\section*{Problem 4-16 (44)}
\label{sec:4-16}
Neglecting ion motions, the dispersion relation for X-waves is given by equation [4-104]
\begin{equation*}
	c^2k^2 = \omega^2 - \omega^2_p\dfrac{\omega^2 - \omega_p^2}{\omega^2 - \omega_h^2},
\end{equation*}
where the group velocity is given by
\begin{equation*}
	v_g = \dfrac{d\omega}{dk}.
\end{equation*}
As such, we differentiate both sides of our first equation to obtain
\begin{equation*}
	2kc^2dk = 2\omega d\omega - \dfrac{2\omega(\omega^2_p - \omega^2_h)}{(\omega^2 - \omega_h^2)^2}\omega_p^2d\omega.
\end{equation*}
Our group velocity is therefore given by
\begin{equation*}
	v_g = \dfrac{kc^2}{\omega\left[1 + \dfrac{\omega_c^2\omega_p^2}{(\omega^2 - \omega_h^2)^2} \right]}.
\end{equation*}
Cutoffs and resonances occur when \(k\to\infty \) and \(k=0\), respectively. Taking \(k\to\infty\) implies \(\omega=\omega_h \), but since \(\left[1 + \dfrac{\omega_c^2\omega_p^2}{(\omega^2 - \omega_h^2)^2} \right] \) is higher order than \(k\) it will go to infinity more quickly than the numerator, and the group velocity will be zero. Taking \(k=0\) gives very clearly a group velocity of zero.

%\section*{Problem 4-17 }
%\label{sec:4-17}
%Requires drawing

\section*{Problem 4-18 (45)}
\label{sec:4-18}
For an L-wave, the dispersion relation is given by
\begin{equation*}
	\dfrac{c^2k^2}{\omega^2} = 1 - \dfrac{\omega_p^2/\omega^2}{1 + \omega_c/\omega}.
\end{equation*}
Cutoff occurs when \(k=0\), implying
\begin{align*}
	0=&1 - \dfrac{\omega_p^2/\omega^2}{1 + \omega_c/\omega} \\
	\dfrac{\omega_p^2}{\omega^2} =& 1 + \dfrac{\omega_c}{\omega} \\
	\omega_p^2 =& \omega^2 + \omega_c\omega \\
	\dfrac{n_0e^2}{m\epsilon_0} =& \omega + \omega\dfrac{eB}{m}\\
	n_0 =& \dfrac{m\epsilon_0}{e^2}\left[2\pi f + 2\pi f\dfrac{eB}{m} \right]\\
	n_0 =& 3.89 \times 10^{17}\text{m}^{-3}.
\end{align*}
Which is our critical density for our L-wave with \(f = 2.8\)GHz and \(B = 0.3\)T.

\section*{Problem 4-19 (46)}
\label{sec:4-19}
Using the dispersion relation for R waves and the definition of the phase velocity we obtain
\begin{equation*}
	\dfrac{c^2k^2}{\omega^2} = c^2v_\phi^{-2} = 1 - \dfrac{\omega_p^2/\omega^2}{1 - \omega_c/\omega}.
\end{equation*}
The maximum can be determined via differentiation
\begin{equation*}
	-2c^2v_\phi^{-3}\dd{v_\phi}{\omega} = \dfrac{\omega_p^2(2\omega - \omega_c)}{\left(\omega^2 - \omega\omega_c\right)^2}.
\end{equation*}
The maximum will occur when \(2\omega - \omega_c \) or \(\omega = \frac{1}{2}\omega_c \), which is what we expected for our Whistler mode. 
Plugging this value into our original equation, we obtain
\begin{equation*}
	\left.\dfrac{c^2}{v^2_\phi}\right|_{\omega = \frac{1}{2}\omega_c} = 1 - \dfrac{4\omega_p^2/\omega_c^2}{1 - 2} = 1 + \dfrac{4\omega_p^2}{\omega_c^2} > 1.
\end{equation*}
Therefore, the maximum phase velocity of the low frequency whistler mode occurs at \(\omega = \frac{1}{2}\omega_c\) and is less than the speed of light. 

\section*{Problem 4-20 (47)}
\label{sec:4-20}
Like in problem 4-16, we will again differentiate both sides of the dispersion relation (although this time for R --- rather than X --- waves) to obtain
\begin{equation*}
	2kc^2dk = 2\omega d\omega - \dfrac{(\omega - \omega_c) - \omega}{(\omega - \omega_c)^2}\omega_p^2d\omega.
\end{equation*}
If \(\omega \ll \omega_c \), we can express our group velocity as
\begin{equation*}
	\dd{\omega}{k} = \dfrac{kc^2}{\omega + \omega_p^2/2\omega_c}.
\end{equation*}
Also, if \(\omega \ll \omega_c \), the dispersion relation is approximately
\begin{equation*}
	c^2k^2 \approx \dfrac{\omega^2_p\omega}{\omega_c} + \omega^2
\end{equation*}
Substituting this into our group velocity expression gives
\begin{equation*}
	v_g = c\dfrac{\sqrt{1 + \omega_p^2/(\omega\omega_c)}}{1 + \omega_p^2/(2\omega\omega_c)}
\end{equation*}
If we assume \(\omega_p^2 \ll \omega\omega_c \), i.e. \(v_\phi^2 \ll c^2 \), then we obtain our desired proportionality
\begin{equation*}
	\dd{\omega}{k} \approx 2c\left(\dfrac{\omega\omega_c}{\omega_p^2}\right)^{\frac{1}{2}} \propto \omega^{\frac{1}{2}}.
\end{equation*}

\section*{Problem 4-21 (48)}
\label{sec:4-21}
Having a positronium plasma implies that, unlike before where the current density was assumed to come entirely as a result of electron motions, we must now update equation [4-82] to be
\begin{equation*}
	\bm{j}_1 = n_0e\left(v_p - v_e\right).
\end{equation*}
From the linearized equation of motion without thermal effects [4-98], we update to include different signs of charge
\begin{equation*}
	v_x = \dfrac{\pm ie}{m\omega}\left(E_x \pm \dfrac{i\omega_c}{\omega}E_y \right)\left(1 - \dfrac{\omega_c^2}{\omega^2}\right)^{-1} 
\end{equation*} 
\begin{equation*}
	v_y = \dfrac{\pm ie}{m\omega}\left(E_y \mp \dfrac{i\omega_c}{\omega}E_x \right)\left(1 - \dfrac{\omega_c^2}{\omega^2}\right)^{-1} 
\end{equation*}
Using these in our wave equation [4-80], while still ignoring the longitudinal term we have
\begin{equation*}
	\left(\omega^2 - c^2k^2 \right)\bm{E}_1 = in_0\omega e(\bm{v}_{e1} - \bm{v}_{p1})/\epsilon_0.
\end{equation*}
Substituting our velocity formulas into our wave equation yields the \(x\)-component of this to be
\begin{equation*}
	\left(\omega^2 - c^2k^2 \right)E_x = \left(\dfrac{-i\omega}{\epsilon_0} \right)(n_0 e)\left(\dfrac{ie}{m\omega} \right)2E_x\left(1 - \dfrac{\omega_c^2}{\omega^2}\right)^{-1} = \dfrac{2\omega_p^2}{1 - \omega_c^2/\omega^2}E_x.
\end{equation*}
Similarly for our \(y\)-component we have
\begin{equation*}
\left(\omega^2 - c^2k^2 \right)E_y = \dfrac{2\omega_p^2}{1 - \omega_c^2/\omega^2}E_y.
\end{equation*}
In either case the dispersion relation is
\begin{equation*}
	\dfrac{c^2k^2}{\omega^2} = 1 - \dfrac{2\omega_p^2}{\omega^2 - \omega_c^2}.
\end{equation*}
This equation --- unlike equation [4-104] --- has a single solution when \(k=0\) given by \(\omega = \sqrt{\omega_c^2 + 2\omega_p^2} \). Therefore the positronium plasma will not experience any Faraday rotation introduced by the multiple cutoff frequencies.

\section*{Problem 4-22 (49)}
\label{sec:4-22}
The angular wavenumber for each of the R- and L-waves will be given by
\begin{equation*}
	k_{R,L} = \dfrac{\omega}{c} \left(1 - \dfrac{\omega_p^2/\omega^2}{1 \pm \omega_c/\omega} \right)^{\frac{1}{2}}.
\end{equation*}
We will assume \(\omega_c \ll \omega \), and expand our square root to obtain
\begin{equation*}
	k_{R,L} = \dfrac{\omega}{c} \left(1 - \dfrac{1}{2}\dfrac{\omega_p^2/\omega^2}{1 \pm \omega_c/\omega} \right).
\end{equation*}
The phase difference between the R and L waves will be given by
\begin{equation*}
	\int_0^L (k_L - k_R)dx = L(k_L-k_R) = \pi,
\end{equation*}
where we have taken the phase difference to be twice the angle that the plane of polarization has rotated. Some algebra yields
\begin{equation*}
	\pi = L(k_L-k_R) = \dfrac{L}{2}\dfrac{\omega}{c}\left(\dfrac{1}{1-\omega_c/\omega} - \dfrac{1}{1+\omega_c/\omega}\right)\dfrac{\omega^2_p}{\omega^2} = L\dfrac{\omega}{c} \dfrac{\omega^2_p\omega_c}{\omega}\dfrac{1}{\omega^2 - \omega_c^2}.
\end{equation*}
Since we are interested in determining the density of the plasma, we rearrange to obtain an expression for \(\omega^2_p \)
\begin{equation*}
	\omega^2_p = \dfrac{\pi}{L}\dfrac{c}{\omega_c}\left(\omega^2 - \omega^2_c \right).
\end{equation*}
Using this equation with the fact that \(\omega^2_p = \frac{n_0e^2}{\epsilon_0m} \), \(\omega = 2\pi\frac{c}{\lambda} = 1.76 \times 10^{10} \)Hz, and \(\omega_c = \frac{qB}{m} = 2.36\times10^{11} \)Hz to obtain
\begin{equation*}
	n = 9.3 \times 10^{17}\text{m}^{-3}
\end{equation*}
Which is the density of the plasma determined via measurement of the rotation of the plane of polarization. 

\section*{Problem 4-23 (50)}
\label{sec:4-23}
Since we have \(\omega^2 \gg \omega_p^2,\omega^2_c \) we can expand the square roots for \(k_{R,L} \) as in the previous problem to obtain
\begin{equation*}
	k_{R,L} = \dfrac{\omega}{c}\left(1 - \dfrac{\omega_p^2/\omega^2}{1 \pm \omega_c/\omega} \right)^{1/2} \approx \dfrac{\omega}{c} \left(1 - \dfrac{1}{2}\dfrac{\omega_p^2/\omega^2}{1 \pm \omega_c/\omega} \right).
\end{equation*}
Again, we take the difference of the phase to be twice the Faraday rotation angle, the difference being given by
\begin{equation*}
	k_L - k_R = \dfrac{\omega}{c}\left[\dfrac{1}{2}\dfrac{\omega_p^2/\omega^2}{1 - \omega_c/\omega}\dfrac{1 + \omega_c/\omega}{1 + \omega_c/\omega} - \dfrac{1}{2}\dfrac{\omega_p^2/\omega^2}{1 - \omega_c/\omega}\dfrac{1 - \omega_c/\omega}{1 - \omega_c/\omega}  \right] = \dfrac{\omega}{2c}\left[\dfrac{2\omega_p^2\omega_c/\omega^3}{1 - \omega_c^2/\omega^2} \right].
\end{equation*}
Since we are assuming \(\omega^2 \gg \omega_c^2  \) we will approximate our denominator to obtain
\begin{equation*}
	k_L-k_R = \dfrac{\omega^2_p\omega_c}{c\omega^2} = \dfrac{1}{c\omega^2}\dfrac{n_0e^2}{m\epsilon_0}\dfrac{eB}{m}.
\end{equation*}
Using this equation, along with the fact that \(\frac{1}{\omega^2} = \left(\frac{\lambda_0}{2\pi c}\right)^2 \) and 
\begin{equation*}
	\int_0^L (k_L - k_R)dx = 2\theta,
\end{equation*}
to obtain
\begin{equation*}
	\theta = \dfrac{1}{2}\left(k_L - k_R\right) = \dfrac{1}{2}\dfrac{\lambda_0^2}{4\pi^2c^3}\dfrac{e^2}{m\epsilon_0}\dfrac{e}{m}B(z)n(z).
\end{equation*}
Evaluating the constants in this equation and converting to degrees yields
\begin{equation*}
	\theta = 1.5\times10^{-11}\lambda_0^2\int_{0}^{L}B(z)n(z)dz,
\end{equation*}
which is exactly the expression for the Faraday rotation we have sought. 

%\section*{Problem 4-24}
%\label{sec:4-24}
%Skip

\section*{Problem 4-25 (51)}
\label{sec:4-25}
The dispersion relationship for X-waves is given by equation [4-104], while the specific cutoff frequency relation is given by equation [4-107]
\begin{equation*}
	\omega^2 \pm \omega\omega_c = \omega_p^2 = \dfrac{n_0e^2}{m\epsilon_0}
\end{equation*}
Rearranging this yields
\begin{equation*}
	n_{cx} = \dfrac{m\omega}{\epsilon_0e^2}(\omega \pm \omega_c),
\end{equation*}
the \(+\) sign corresponds to the higher density L wave cutoff while the \(-\) sign corresponds to the lower density R wave cutoff. If we were to take a \(v^2_\phi/\cal^2 \) vs. \(\omega \) diagram like the one shown in Figure [4-36] in the text, we can say that by using extraordinary waves with \(\omega < \omega_c \) one can avoid the cutoff range between \(\omega_h \) and \(\omega_R \).

\section*{Problem 4-26 (52)}
\label{sec:4-26}
The Alfv\'en velocity is given by
\begin{equation*}
	v_A = \dfrac{B}{\sqrt{\mu_0\rho}} = 8.92 \times 10^{-6}\text{m/s}.
\end{equation*}
Since the Alfv\'en velocity represents a phase velocity is can be larger than \(c\), but since no information is propagating this does not imply that anything is travelling through space faster then light. 

\section*{Problem 4-27 (53)}
\label{sec:4-27}
Since we have \(\rho = nM \), we can say that our Alfv\'en velocity is 
\begin{equation*}
v_A = \dfrac{B}{\sqrt{\mu_0nM_H}} = 2.183 \times 10^{4}\text{m/s}.
\end{equation*}

\section*{Problem 4-28 (54)}
\label{sec:4-28}
In order to determine the region of the CMA diagram were the experiment is located we must determine \(\dfrac{\omega_c}{\omega} \) and \(\dfrac{\omega_p^2}{\omega^2} \). We have 
\begin{equation*}
	\omega_p = 1.79 \times 10^{9}\text{Hz}, \quad \omega_p = 1.76 \times 10^{9}\text{Hz}, \quad \omega = 1.00 \times 10^{9}\text{Hz}.
\end{equation*}
Since this gives both \(\dfrac{\omega_c}{\omega} > 1 \) and \(\dfrac{\omega_p^2}{\omega^2} > 1\), we must determine whether we are operarting above or below the L-cutoff line. To do so, we take 
\begin{equation*}
	\omega_L  =\frac{1}{2}\left[-\omega_c + \left(\omega_c^2 + 4\omega_p^2\right)^{1/2} \right] = 1.08 \times 10^{9}\text{Hz}.
\end{equation*}
Since we have \(\omega_L > \omega \) we can place our experiment on the following CMA diagram, marked by \(\ast \).
\\
\begin{figure}[H]
\centering	
	\begin{tikzpicture}
		\draw (0,0) -- (8,0);
		\draw (0,0) -- (0,8);
		\draw (3,0) -- (3,8);
		\draw (0,3) -- (8,3);
		\draw (0,3) -- (3,0);
		\draw (0,3) to[out = 350, in = 100] (3,0);
		\node at (-0.5,5) {\(\dfrac{\omega_c}{\omega}\)};
		\node at (5,-0.5) {\(\dfrac{\omega_p^2}{\omega^2}\)};
		\node at (-0.5,3) {\(1\)};
		\node at (3,-0.5) {\(1\)};
		\draw (3,0) to[out = 30, in = 270] (5,3);
		\draw (5,3) to[out = 90, in = 330] (3,7);
		\node at (6,6) [circle,draw,inner sep=0pt,minimum width=0.75cm] {\(\ast \)};
	\end{tikzpicture}
	\caption{CMA Diagram, with mark for experiment}
\end{figure}
Based on this, the only electromagnetic waves that will propagate though this plasma are the whistler modes. 

\section*{Problem 4-29 (55)}
\label{sec:4-29}
Having a standing wave with maximum amplitude at the mid-plane and nodes at the edges of the plasma column requires a wavelength of \(\lambda = 2L \), where \(L\) is the radius of the cylindrical plasma column. Using the definition of the angular wavenumber, Alfv\'en velocity, and ion cyclotron frequency, respectively, we obtain
\begin{equation*}
	L = \dfrac{\lambda}{2} = \dfrac{\pi}{k} = \dfrac{\pi v_A}{\omega} = \dfrac{\pi v_A}{0.1\Omega_c} = \dfrac{\pi B M}{0.1eB\sqrt{\mu_0nM}} = 2.26\text{m}.
\end{equation*}
If the Q-machine were used instead we would have
\begin{equation*}
	L =  \dfrac{\pi B M}{0.1eB\sqrt{\mu_0n'M'}} = 82\text{m}.
\end{equation*}

\section*{Problem 4-30 (56)}
\label{sec:4-30}
Ignoring the interstellar magnetic field, our dispersion relation is given by equation [4-85]
\begin{equation*}
	\omega^2 = \omega_p^2 + c^2k^2 \implies \dfrac{c^2k^2}{\omega^2} = 1 - \dfrac{\omega_p^2}{\omega^2}.
\end{equation*}
Differentiating both sides of the first equation yields
\begin{equation*}
	2\omega d\omega = 2kc^2dk \implies v_g = c\left(1 - \dfrac{\omega_p^2}{\omega^2}\right)^{1/2} \approx c\left(1 - \dfrac{\omega_p^2}{2\omega^2}\right).
\end{equation*}
Our approximation is valid as long as \(\omega^2 \gg \omega_p^2 \). Since \(t = x/v_g \), we can obtain an expression for \(\dd{t}{\omega} \) by differentiating the inverse of the group velocity like so
\begin{equation*}
	\dd{t}{\omega} = \dfrac{x}{c}\left(1 - \dfrac{\omega_p^2}{\omega^2}\right)^{-2}\left(-\dfrac{\omega_p^2}{\omega^2} \right) \approx -\dfrac{x}{c}\dfrac{\omega_p^2}{\omega^3}.
\end{equation*}
Taking the recipricol implies that 
\begin{equation*}
	\dd{f}{t} \approx -\dfrac{c}{x}\dfrac{f^3}{f^2_p}.
\end{equation*}
In order to determine the distance of the pulsar, we have
\begin{equation*}
	x = \dfrac{cf^3}{f^2_p}\left(-\dd{f}{t} \right)^{-1} = 63\text{ parsecs}.
\end{equation*}


%\section*{Problem 4-31}
%\label{sec:4-31}
%Looks doable, but quite long. Maybe come back

\section*{Problem 4-32 (57)}
\label{sec:4-32}
In order to show that the time-averaged electron kinetic energy per unit volume is equal to the electric field energy density we will first note the definition of the time-averaged electron kinetic energy per unit volume to be
\begin{equation*}
	\EE = n_0\frac{m}{2}\left<v_e^2 \right>.
\end{equation*}
Using the fact that based on our equation of motion we have
\begin{equation*}
	v_e = \dfrac{e}{im\omega}E \implies \left<v_e^2\right> = \dfrac{e^2}{m^2\omega^2}\left<E^2\right>.
\end{equation*}
Substituting this into our first equation we obtain
\begin{equation*}
	\EE = n_0\frac{m}{2} \dfrac{e^2}{m^2\omega^2}\left<E^2\right> = \epsilon_0\dfrac{\omega^2_p}{2\omega^2}\left<E^2\right>.
\end{equation*}
However since for our Langmuir plasma we have \(\omega_p = \omega \), we can therefore say that 
\begin{equation*}
	\EE = \dfrac{\epsilon_0}{2}\left<E^2\right>,
\end{equation*}
as expected.


\section*{Problem 4-33 (58)}
\label{sec:4-33}
For Alfv\'en waves if we assume \(\Omega^2_c \gg \omega^2 \) then we can say that 
\begin{equation*}
	v_i \approx -\dfrac{E_1}{B_0}, \implies \EE = \dfrac{1}{2}Mn_0\dfrac{\left<E_1^2\right>}{B_0}.
\end{equation*}
Furthermore by Faraday's law we have \(E_1 = \dfrac{\omega}{k}B_1 \), therefore
\begin{equation*}
	\EE = \dfrac{Mn_0}{2B_0}\dfrac{\omega^2}{k^2}\left<B^2_1 \right>.
\end{equation*}
Finally, we note that for an Alfv\'en wave we have \(\dfrac{\omega^2}{k^2} = \dfrac{B_0}{\mu_0n_0M} \), which implies that
\begin{equation*}
	\EE = \dfrac{\left<B_1^2\right>}{2\mu_0}.
\end{equation*}

%\section*{Problem 4-34}
%\label{sec:4-34}
%Since we have cutoff for our L-waves occurring at \(\omega = \omega_L < \omega_p \) \\
%MAYBE COME BACK TO THIS ONE.
%
%\section*{Problem 4-35}
%\label{sec:4-35}
%MIGHT NOT BE ABLE TO DO THIS ONE\\
%NEXT PROBLEM MAY HELP HERE

\section*{Problem 4-36 (59)}
\label{sec:4-36}
By including the \(\nabla p \) term, our linearized equations of motion for both species are updated to become
\begin{equation*}
	-i\omega m n_0 \bm{v}_1 = \pm e \left(\bm{E} + \bm{v}_1 \times \bm{B}_0 \right) - \gamma k_BT\bm{k}n_1.
\end{equation*}
Multiplying both sides of our equation of motion with \(\bm{k} \cdot  \) yields
\begin{equation*}
-i\omega m n_0 \bm{k} \cdot\bm{v}_1 = \pm e \left(\bm{k} \cdot\bm{E} + \bm{k} \cdot\bm{v}_1 \times \bm{B}_0 \right) - \gamma k_BTk^2n_1.
\end{equation*}
However, for transverse waves we have \(\bm{k} \cdot \bm{E} = 0 \); we also have \(\bm{k} \cdot \left(\bm{v}_1 \times \bm{B}_0\right) = \bm{v}_1 \cdot \left(\bm{k} \times \bm{B}_0\right) = 0 \), by the problem specification. Our equation of motion therefore becomes 
\begin{equation*}
	\omega m n_0 i \left(\bm{k} \cdot \bm{v}_1 \right) = -\gamma k_BTk^2n_1.
\end{equation*}
We can use the continuity equation,
\begin{equation*}
	n_0 i \left(\bm{k} \cdot \bm{v}_1 \right) = i\omega n_1,
\end{equation*}
to obtain
\begin{equation*}
	i\omega^2m n_1 = -\gamma k_BTk^2n_1.
\end{equation*}
We see that the dependence on \(n_1\) (and therefore \(\nabla p \)) vanishes, and transverse waves will be unaffected by the \(\nabla p \) term.

\section*{Problem 4-37 (60)}
\label{sec:4-37}
The linearized equation of motion for the electron is now given by
\begin{equation*}
	i\omega m \bm{v}_1 = -e\bm{E}_1 - m \bm{v}_1\nu.
\end{equation*}
Solving for \(\bm{v}_1 \) and substituting this into our equation for \(\bm{j}_1 \) yields 
\begin{equation*}
	\bm{v}_1 = \dfrac{-e\bm{E}_1}{i\omega m + m\nu} \implies \bm{j}_1 = -\dfrac{n_0e^2\bm{E}_1}{im(\omega+i\nu)}.
\end{equation*}
Using this expression for \(\bm{j}_1 \) in our transverse wave equation we obtain
\begin{equation*}
	\left(\omega^2 - c^2k^2 \right)\bm{E}_1 = \dfrac{\omega}{\omega + i\nu}\omega_p^2\bm{E}_1.
\end{equation*}
Dividing through by \(\omega^2 \) yields the expected dispersion relation
\begin{equation*}
	\dfrac{c^2k^2}{\omega^2} = 1 - \dfrac{\omega^2_p}{\omega(\omega + i\nu)}.
\end{equation*}

(b) If we assume \(\nu/\omega \ll 1 \) then we can say that 
\begin{equation*}
	\left(1 + i\dfrac{\nu}{\omega} \right)^{-1} \approx 1 - i\dfrac{\nu}{\omega},
\end{equation*}
and therefore update our dispersion relation to 
\begin{equation*}
	c^2k^2 \approx \omega^2 - \omega_p^2 + i\nu\dfrac{\omega^2_p}{\omega}.
\end{equation*}
In order to determine our \(\gamma \equiv -\text{Im}(\omega) \), we will assume \(\omega = x + iy \), substituting this into our dispersion relation yields
\begin{equation*}
	c^2k^2 + \omega_p^2 = x^2 - y^2 + 2xyi + i\dfrac{\omega^2_p\nu}{x+iy} = x^2 - y^2 + \dfrac{\omega_p^2\nu}{x^2+y^2} + i\left(2xy + \dfrac{\omega_p^2\nu x}{x^2+y^2} \right).
\end{equation*}
Since the right hand side of our equation is real, we must have the imaginary portion of our left hand side being zero, this is to say
\begin{equation*}
	2xy =- \dfrac{\omega_p^2\nu x}{x^2+y^2} \implies  \text{Im}(\omega) \equiv y = - \dfrac{\omega_p^2\nu}{2|\omega|^2}.
\end{equation*}
This is the damping rate relation that is valid when \(\nu/\omega \ll 1 \).\\

Lastly, by again applying the approximation from part (b), we obtain
\begin{equation*}
	k^2 \approx \dfrac{\omega^2 - \omega_p^2}{c^2} + i\nu\dfrac{\omega^2_p}{\omega c^2}.
\end{equation*}
We will follow the same procedure as part (b) and assume \(k = x + iy \), and \(k^2 = x^2 - y^2 + 2xyi \), this yields 
\begin{equation*}
	x^2 - y^2 + 2xyi \approx \dfrac{\omega^2 - \omega^2_p}{c^2} + i\nu\dfrac{\omega^2_p}{\omega c^2},
\end{equation*}
each part of which is an equation that can be used to solve for one of the two unknowns (\(x\) and \(y\)). We have
\begin{equation*}
	x^2 = \dfrac{\omega^2 - \omega^2_p}{c^2} + y^2, \quad \& \quad x^2y^2 = \left(\dfrac{\omega^2_p\nu}{2\omega c^2}\right)^2.
\end{equation*}
Combining these two equations yields the quartic equation
\begin{equation*}
	\dfrac{\omega^2 - \omega^2_p}{c^2}y^2 + y^4 = \left(\dfrac{\omega^2_p\nu}{2\omega c^2}\right)^2,
\end{equation*}
substituting \(z = y^2\), yields a quadratic equation, which can be solved for \(y\). Determining this solution using Maple gives
\begin{equation*}
	\delta = \left(\text{Im}(k)\right)^{-1} = \dfrac{2c}{\nu}\dfrac{\omega^2}{\omega_p^2}\left(1 - \dfrac{\omega^2_p}{\omega^2}\right)^{1/2}.
\end{equation*}


\section*{Problem 4-38 (61)}
\label{sec:4-38}
As in the previous problem each electron will have collision and electric field forces acting upon it, yield a linearized equation of motion
\begin{equation*}
i\omega m \bm{v}_1 = -e\bm{E}_1 - m \bm{v}_1\nu,
\end{equation*}
where \(\nu = n_n\overline{\sigma v} =100\text{s}^{-1}\). Analogously to the previous question, our dispersion relation is given by
\begin{equation*}
	k^2 = \dfrac{\omega^2 - \omega^2_p}{c^2} + i\dfrac{\omega^2_p\nu}{\omega^2_c}.
\end{equation*}
The electric field for a microwave plasma can be described by the equation
\begin{equation*}
	E = E_0\exp\{i\left(\text{Re}(k)r - \omega t \right) \}\exp\{-\text{Im}(k)r \}.
\end{equation*}
Again, like the previous question, the term \(\exp\{-\text{Im}(k)r \} \) represents the attenuation as the wave travells through the ionosphere. In order to determine the amount of attenuation, we take the ratio of magnitudes of the wave before and after traveling 100km 
\begin{equation*}
	\dfrac{E(100\text{km})}{E(0\text{km})} = \exp\{-2\text{Im}(k)(100\text{km}) \}.
\end{equation*}
We note that 
\begin{equation*}
	\omega^2_p = 3.17\times 10^{14}\text{s}^{-1}, \quad \text{Im}(k) = \dfrac{\nu}{2c}\dfrac{\omega^2_p}{\omega^2}\left(1 - \omega^2_p/\omega^2\right)^{-1/2} = 1.34\times 10^{12}\text{m}^{-1}.
\end{equation*}
This yields 
\begin{equation*}
	\dfrac{E(100\text{km})}{E(0\text{km})} = \exp(-2.70\times 10^{-7}) = 0.99999
\end{equation*}
We therefore say that almost no beam power is lost to heating up the ionosphere. 


%\section*{Problem 4-39}
%\label{sec:4-39}
%Too much algebra

\section*{Problem 4-40 (62)}
\label{sec:4-40}
In the first situation, when \(\bm{E} \) is in the \(\hat{z}\)-direction, we have \(\bm{E}_1\parallel\bm{B}_0 \). This is an ordinary wave, the dispersion relation for which is given by
\begin{equation*}
	\omega^2 = \omega^2_p + c^2k^2 \implies k^2 = \left(\dfrac{2\pi}{c\lambda_0} \right)^2 - \dfrac{n_0e^2}{m\epsilon_0c^2} \implies k = 543.7\text{m}^{-1}.
\end{equation*}
We can therefore say that the number of wavelengths in the waveguide is
\begin{equation*}
	N = \dfrac{d}{\lambda} = \dfrac{dk}{2\pi} = 8.6
\end{equation*}

In the second situation we have \(\bm{E}_1\perp\bm{B}_0 \) --- an extraordinary wave --- the dispersion relation for which is given by
\begin{equation*}
	\dfrac{c^2k^2}{\omega^2} = 1 - \dfrac{\omega^2_p}{\omega^2}\dfrac{\omega^2 - \omega^2_p}{\omega^2 - \omega^2_p - \omega^2_c}.
\end{equation*}
Where \(\omega\) and \(\omega_p^2 \), are the same as the previous situation, solving for \(k\) gives, \(k = 832.2\text{m}^{-1} \). Based on this we can say that there are
\begin{equation*}
	N =\dfrac{d}{\lambda} = 13.2
\end{equation*}
wavelengths in the waveguide.

\section*{Problem 4-41 (63)}
\label{sec:4-41}
Since we have no electric field or magnetic field, and no velocity in our equilibrium state, we have 
\begin{equation*}
	\bm{B}_0 = 0, \bm{E}_0 = 0, T_e = 0.
\end{equation*}
The relevant Maxwell equations yield, once more, equation [4-80]
\begin{equation*}
	(\omega^2 - c^2k^2 )\bm{E}_1 = -i\omega\bm{j}_1/\epsilon_0.
\end{equation*}
We must now update our equation for \(\bm{j}_1 \) to account for our non-hydrogen ions. To do so, we take
\begin{equation*}
	\bm{j}_1 = n_{0+}Ze\bm{v}_{0+} - n_{0-}e\bm{v}_{0-} \implies \bm{j}_1 = Zn_{0+}e\left(\bm{v}_{0+} - \bm{v}_{0-}\right)
\end{equation*}
Our linearized equations of velocity are
\begin{equation*}
	iM_+\omega\bm{v}_{0+} = -Ze\bm{E}_1,
\end{equation*}
\begin{equation*}
	iM_-\omega\bm{v}_{0-} = e\bm{E}_1.
\end{equation*}
Using these along with equation [4-80] and our expression for the current density, we obtain
\begin{equation*}
	\bm{j}_1 = -Zn_{0+}e\left(\dfrac{Ze\bm{E}_1}{iM_+\omega} + \dfrac{e\bm{E}_1}{iM_-\omega}\right) = \dfrac{Zn_{0+}e^2}{i\omega}\left(\dfrac{Z}{M_+} + \dfrac{1}{M_-}\right)\bm{E}_1.
\end{equation*}
This yields the following dispersion relation:
\begin{equation*}
	\omega^2 - c^2k^2 = \dfrac{Z^2e^2n_{0+}}{M_+\epsilon_0} + \dfrac{e^2n_{0-}}{M_-\epsilon_0}.
\end{equation*}

\section*{Problem 4-42 (64)}
\label{sec:4-42}
The Boltzmann relation and plasma approximation state that, respectively, 
\begin{equation*}
	n_{e1} \approx n_0\dfrac{e\phi_1}{k_BT_e}, \quad Zn_{A1} + n_{H1} = n_{e1}.
\end{equation*}
For our problem, the linearized equations of motion are
\begin{equation*}
	-iM_A\omega\bm{v}_{A1} = Ze\left(-ik\phi_1\right),
\end{equation*}
\begin{equation*}
-iM_H\omega\bm{v}_{H1} = e\left(-ik\phi_1\right).
\end{equation*}
While the continuity equations are
\begin{equation*}
	i\omega n_{A1} = n_{A}ikv_{A1},
\end{equation*} 
\begin{equation*}
	i\omega n_{H1} = n_{H}ikv_{H1}.
\end{equation*}
Combining the equations of motion and continuity, along with out Boltzmann and plasma approximations we get 
\begin{equation*}
	\dfrac{\omega^2}{k^2} = \dfrac{k_BT_e}{n_0}\left(\dfrac{Z^2n_A}{M_A} + \dfrac{n_H}{M_H} \right).
\end{equation*}
The phase velocity is given by
\begin{equation*}
	v_\phi = \dfrac{\omega}{k} = \left(\dfrac{k_BT_e}{n_0}\left(\dfrac{Z^2n_A}{M_A} + \dfrac{n_H}{M_H} \right)\right)^{1/2}.
\end{equation*}

\section*{Problem 4-43 (65)}
\label{sec:4-43}
We wish to use the Poisson equation
\begin{equation*}
	\epsilon_0\nabla\cdot E_1 = n_{+1}e - Zn_{-1}e,
\end{equation*}
the continuity equations
\begin{equation*}
	n_{+1} = \dfrac{k}{\omega}n_0v_{+1},
\end{equation*}
\begin{equation*}
n_{-1} = \dfrac{k}{\omega}n_0v_{-1},
\end{equation*}
and the equations of motion, which in the absence of collisions, magnetic fields, and thermal motions are 
\begin{equation*}
	v_{+1} = \dfrac{-eE_1}{i\omega M_+},
\end{equation*}
\begin{equation*}
	v_{-1} = \dfrac{ZeE_1}{i\omega M_-},
\end{equation*}
to derive a dispersion relation for the plasma of antifermium nuclei at a remote part of the universe. Substituting our expressions for velocity into our continuity equations, then the resulting expressions for density into Poisson's equation we obtain
\begin{equation*}
	\epsilon_0(ik)\bm{E}_1 = \dfrac{k}{\omega}n_0\dfrac{-e^2\bm{E}_1}{i\omega M_+} - Z\dfrac{k}{\omega}n_0\dfrac{Ze^2\bm{E}_1}{i\omega M_-}.
\end{equation*}
Rearranging this we obtain
\begin{equation*}
	\omega^2 = \omega_{p+}^2 + \omega_{p-}^2.
\end{equation*}
Where we have defined the plasma frequency for the antifermium to be 
\begin{equation*}
	\omega^2_{p-} = \dfrac{Z^2n_0e^2}{M_-\epsilon_0},
\end{equation*}
and the positron plasma frequency to be 
\begin{equation*}
	\omega^2_{p+} = \dfrac{n_0e^2}{M_+\epsilon_0},
\end{equation*}

\section*{Problem 4-44 (66)}
\label{sec:4-44}
If we assume the signal is being cutoff from below, then the highest possible cutoff frequency is \(\omega = \omega_R \) which will satisfy
\begin{equation*}
	\omega^2 - \omega\omega_c - \omega^2_p = 0 \implies \dfrac{n_0e^2}{\epsilon_0m} = 4\pi^2f^2 - 2\pi f\dfrac{eB}{m}.
\end{equation*}
Solving for \(n_0\), we obtain
\begin{equation*}
	n = 2.8\times 10^{13}\text{m}^{-1},
\end{equation*}
which gives a lower limit to the plasma density in the Crab nebula.

\section*{Problem 4-45 (67)}
\label{sec:4-45}
By assuming that the atmosphere of Jupiter contains cold, single charged ions, we can take \(k_BT_i = 0 \). The sonic velocty is then
\begin{equation*}
	v_s = \left(\dfrac{k_BT_e}{M} \right)^{1/2},
\end{equation*}
while the Alfv\'en velocity is 
\begin{equation*}
	v_A = \dfrac{B}{\sqrt{\mu_0Mn_0}}.
\end{equation*}
Since the instruments on our spacecraft indicate that we have acoustic shock waves but not magnetic shock waves, we know that
\(v > v_s \), this is to say that
\begin{equation*}
	v^2 > \dfrac{k_BT_e}{M} \implies T_e < \dfrac{Mv^2}{k_B} = 1.2.2\times 10^7\text{K}.
\end{equation*}
Likewise, since we are not detecting any magnetic shock waves we know that \(v < v_A \)
\begin{equation*}
	v^2 < \dfrac{B^2}{\mu_0Mn_0} \implies n_0 < \dfrac{B}{\mu_0Mv^2} = 4.76\times 10^{11}\text{m}^{-3}.
\end{equation*}
These inequalities give the upper limits on temperature and density.

\section*{Problem 4-46 (68)}
\label{sec:4-46}
If we let \(\omega = \omega_R \) and \(\omega = \omega_h \) at \(r_1\) and \(r_2\) respectively, then we have, the the definitions of \(\omega_R\) and \(\omega_h \)
\begin{equation*}
	\omega_{p1}^2 = \omega^2 - \omega\omega_c
\end{equation*}
\begin{equation*}
	\omega_{p2}^2 = \omega^2 - \omega_c^2.
\end{equation*}
Taking the difference of these two equations yields
\begin{equation*}
	\omega_{p2}^2 - \omega_{p1}^2 = \dfrac{e^2}{\epsilon_0m}\left(n_2 - n_1\right) = \omega_c\left(\omega - \omega_c \right),
\end{equation*}
or
\begin{equation*}
	n_2 - n_1 = \dfrac{\epsilon_0 m}{e^2}\left(\omega - \omega_c\right)\omega_c.
\end{equation*}
If we approximate our difference with a derivative, we obtain
\begin{equation*}
	n_2 - n_1 \approx d\left|\pp{n}{r}\right|.
\end{equation*}
using the approximation provided in the problem statement gives
\begin{equation*}
	n_2 - n_1 \approx d\left|\pp{n}{r}\right| \approx d\dfrac{n_1}{r_0}.
\end{equation*}
From our definition of \(\omega_{p1}^2 = \dfrac{n_1e^2}{\epsilon_0m} = \omega(\omega - \omega_c) \), we can substitute for \(n_1\) as follows
\begin{equation*}
	d\dfrac{n_1}{r_0} = \dfrac{\epsilon_0m}{e^2}\omega(\omega - \omega_c)\dfrac{d}{r_0}.
\end{equation*}
What we can say now is that
\begin{align*}
	n_2 - n_1 \approx& \;\dfrac{\epsilon_0 m }{e^2}\omega(\omega - \omega_c)\dfrac{d}{r_0}\\
	\dfrac{e^2}{\epsilon_0 m }(n_2 - n_1) \approx& \;\omega(\omega - \omega_c)\dfrac{d}{r_0}\\
	\omega_c(\omega - \omega_c) \approx& \;\omega(\omega - \omega_c)\dfrac{d}{r_0}\\
	d =& \dfrac{\omega_c}{\omega}r_0.
\end{align*}
This is to say that the length of the evanescent layer is given by \(d = \dfrac{\omega_c}{\omega}r_0.\)

%\section*{Problem 4-47}
%\label{sec:4-47}
%Easy, but requires drawing
%
%\section*{Problem 4-48}
%\label{sec:4-48}
%Seems tough, and so do 49, 50, 51
