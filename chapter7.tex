\chapter{Kinetic Theory}
\label{ch:Seven}

\section*{Problem 7-1 (86)}
\label{sec:7-1}
The Landau damping rate is given by
\begin{equation*}
	\text{Im}(\dfrac{\omega}{\omega_p}) = -0.22\sqrt{\pi}\left(\dfrac{\omega_p}{kv_\text{th}} \right)^3\exp\left(\dfrac{-1}{2k^2\lambda_D^2}\right).
\end{equation*}
We have
\begin{equation*}
	\omega_p = \left(\dfrac{ne^2}{m\epsilon_0}\right)^{1/2} = 17.84 \text{GHz},
\end{equation*}
\begin{equation*}
	v_\text{th} = \left(\dfrac{2k_BT}{m}\right)^{1/2} = 1.876\times 10^6 \text{m/s},
\end{equation*}
\begin{equation*}
	\lambda_D^2 = \dfrac{\epsilon_0k_BT}{ne^2} = 5.53 \times 10^{-9}\text{m}^{2}.
\end{equation*}
Combining all of these we obtain
\begin{equation*}
	\left|\text{Im}\left(\dfrac{\omega}{\omega_p}\right)\right| = 0.1358.
\end{equation*}

\section*{Problem 7-2 (87)}
\label{sec:7-2}
This problem is quite similar to the previous one, we first require the Landau damping rate, given by
\begin{equation*}
\text{Im}(\dfrac{\omega}{\omega_p}) = -0.22\sqrt{\pi}\left(\dfrac{\omega_p}{kv_\text{th}} \right)^3\exp\left(\dfrac{-1}{2k^2\lambda_D^2}\right).
\end{equation*}
We have
\begin{equation*}
	k = \dfrac{2\pi}{\lambda} = 628.3\text{m}^{-1},
\end{equation*}
\begin{equation*}
\omega_p = \left(\dfrac{ne^2}{m\epsilon_0}\right)^{1/2} = 1.784 \text{GHz},
\end{equation*}
\begin{equation*}
v_\text{th} = \left(\dfrac{2k_BT}{m}\right)^{1/2} = 1.876\times 10^6 \text{m/s},
\end{equation*}
\begin{equation*}
\lambda_D^2 = \dfrac{\epsilon_0k_BT}{ne^2} = 5.52 \times 10^{-7}\text{m}^{2}.
\end{equation*}
These give a Landau damping factor of
\begin{equation*}
	\left|\text{Im}\left(\dfrac{\omega}{\omega_p}\right)\right| = 0.1363, \implies \left|\text{Im}\left(\omega\right)\right| = 0.243\text{GHz}.
\end{equation*}
In order to determine when the amplitude falls by a factor \(e\), we note that
\begin{equation*}
	\dfrac{E(t)}{E(0)} = e^{-|\text{Im}(\omega)|t},
\end{equation*}
and that this ratio is equal to \(1/e \) when \( t = \frac{1}{|\text{Im}(\omega)|} \).
We therefore sat that after the excitation is removed, it takes \(t = 4.115\)ns for the amplitude to fall by a factor of \(e\).

\section*{Problem 7-3 (88)}
\label{sec:7-3}
The general expression for a Maxwellian distribution centered at \(u_0\) is given by equation [1-2]
\begin{equation*}
	f(u) = n\sqrt{\dfrac{m}{2\pi k_BT}}\exp\left(-\dfrac{1}{2}m\left(u - u_0\right)^2/k_BT \right).
\end{equation*}
Knowing this, the expressions for \(f_p(v) \) and \(f_b(v)\) are
\begin{equation*}
	f_p(v) = \dfrac{n_p}{a\sqrt{\pi}}\exp\left(-\dfrac{v^2}{a^2} \right),
\end{equation*}
\begin{equation*}
	f_b(v) = \dfrac{n_b}{b\sqrt{\pi}}\exp\left(-\dfrac{(v-V)^2}{b^2} \right).
\end{equation*}
In order to determine the phase velocity \(v_\phi \) we must optimize \(f'_b(v)\), this requires the first two derivatives of \(f_b\) with respect to \(v\)
\begin{equation*}
	f_b'(v) = \dfrac{n_b}{b\pi^{1/2}}\dfrac{-2(v-V)}{b^2}\exp\left(-\dfrac{(v-V)^2}{b^2} \right),
\end{equation*}
\begin{equation*}
	f_b''(v) = \dfrac{-2n_b}{b^3\pi^{1/2}}\left[1 - \dfrac{2(v-V)}{b^2}\right]\exp\left(-\dfrac{(v-V)^2}{b^2} \right).
\end{equation*}
Setting the second derivative to zero requires
\begin{equation*}
	v - V = \pm \dfrac{b}{\sqrt{2}}, \implies v_\phi = V - \dfrac{b}{\sqrt{2}}.
\end{equation*}
This gives a value of the largest positive slope to be
\begin{equation*}
	f_b'(v_\phi) = \left(\dfrac{2}{\pi}\right)^{1/2}\dfrac{n_b}{b^2}e^{-1/2}.
\end{equation*}
Finding \(f_p'(v_\phi)\) is very straightforward 
\begin{equation*}
	f_p'(v) = \dfrac{n_p}{a\pi^{1/2}}\dfrac{-2v}{a^2}\exp\left(-\dfrac{v^2}{a^2} \right),
\end{equation*}
\begin{equation*}
	f_p'(v_\phi) = \dfrac{n_p}{a\pi^{1/2}}\dfrac{-2}{a^2}\left(V - \dfrac{b}{\sqrt{2}}\right)\exp\left(-\dfrac{\left(V - b/\sqrt{2}\right)^2}{a^2} \right).
\end{equation*}
Assuming \(V \gg b\), our relation become
\begin{equation*}
	f_p'(v_\phi) = \dfrac{-2n_p}{a^3\pi^{1/2}}V\exp\left(-\dfrac{V^2}{a^2} \right).
\end{equation*}
Setting the slope of the total beam distribution function equal to zero gives
\begin{equation*}
	f_p'(v_\phi) + f_b'(v_\phi) = 0 \implies \left(\dfrac{2}{\pi}\right)^{1/2}\dfrac{n_b}{b^2}e^{-1/2} = \dfrac{2n_p}{a^3\pi^{1/2}}V\exp\left(-\dfrac{V^2}{a^2} \right).
\end{equation*}
We need now simply solve for the ratio of the beam to plasma densities
\begin{equation*}
	\dfrac{n_b}{n_p} = \sqrt{2e}\dfrac{b^2}{a^3}V\exp\left(-\dfrac{V^2}{a^2} \right).
\end{equation*}
By noting that \(\dfrac{b^2}{a^2} = \dfrac{T_b}{T_p} \), we obtain our desired relation
\begin{equation*}
	\dfrac{n_b}{n_p} = \sqrt{2e}\dfrac{T_b}{T_p}\dfrac{V}{a}\exp\left(-\dfrac{V^2}{a^2} \right).
\end{equation*}

\section*{Problem 7-8 (89)}
\label{sec:7-8}
We will assume \(k^2\lambda^2_D \ll 1 \) and invoke equation [7-128] to obtain
\begin{equation*}
	\sum_j\dfrac{n_{0j}}{n_{0e}}Z'\left(\dfrac{\omega}{kv_{\text{th}j}}\right) = \dfrac{2T_i}{T_e}.
\end{equation*}
Since we have an argon plasma with a hydrogen impurity, we assume
\begin{equation*}
	\dfrac{n_{0A}}{n_{0e}} = 1-\alpha \approx 1, \quad \dfrac{n_{0H}}{n_{0e}} = \alpha.
\end{equation*}
Doubling the Landau damping rate requires
\begin{equation*}
	\text{Im}Z'(\zeta_A) = \text{Im}Z'(\zeta_H), \text{ where } \text{Im}Z'(\zeta_j) = -2i\sqrt{\pi}\zeta_j\exp\left(-\zeta_j^2\right).
\end{equation*}
More explicitly, we say that doubling the Landau damping rate requires
\begin{equation*}
	\zeta_A\exp\left(-\zeta_A^2\right) = \alpha \zeta_H\exp\left(-\zeta_H^2\right),
\end{equation*}
solving for \(\alpha\) yields
\begin{equation*}
	\alpha = \dfrac{\zeta_A}{\zeta_H}\exp\left(-(\zeta_A^2 - \zeta_H^2\right)).
\end{equation*}
We note that since the atomic mass of argon is 40, we can say that
\begin{equation*}
\alpha = \sqrt{40}\exp\left(-(\zeta_A^2(1 - 1/40)\right)).
\end{equation*}
We now solve for \(\zeta_A \),
\begin{equation*}
	\zeta_A^2 = \dfrac{13}{2},
\end{equation*}
and determine that a hydrogen impurity density fraction of 
\begin{equation*}
	\alpha = \sqrt{40}\exp(-6.5(0.975)) = 0.012
\end{equation*}
will double the damping rate for the singly ionized argon plasma wave.


