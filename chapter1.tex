\chapter*{Chapter One}
\label{ch:One}

\section*{Problem 1-1}
\label{sec:1-1}
An ideal gas will satisfy the ideal gas equation, with pressure \(P\), volume \(V\), temperature \(T\), Boltzmann's constant \(k_B\), and \(N\) particles. We can obtain the density in units of \(m^{-3}\) by taking.
\begin{align}
	PV &= Nk_BT \\
	n &\equiv \dfrac{N}{V} = \dfrac{P}{k_BT} \\
	  &= \dfrac{101.35\text{ kPa}}{(1.38\times10^{-23}\frac{\text{ J}}{\text{ K}})(273.15\text{ K})}\\
	  &= 2.6887\times10^{25}\text{ m}^{-3}.
\end{align}
Which is in good agreement with the Loschmidt number, as expected. 
For part b), we apply the same reasoning to obtain
\begin{align}
	n &= \dfrac{0.1333\text{ kPa}}{(1.38\times10^{-23}\frac{\text{ J}}{\text{ K}})(293.15\text{ K})}\\
	&= 3.295\times10^{19}m^{-3}.
\end{align}

\section*{Problem 1-2}
\label{sec:1-2}
We seek a constant, \(A\), such that, our one-dimensional Maxwell distribution is normalized. This is to say that we solve the following equation for \(A\):
\begin{align}
	1 &= \int_\infty^\infty \hat{f}(u)du = \int_\infty^\infty A \exp{(-mu^2/2k_BT)}
\end{align}
A common exponential definite integral is
\begin{align}
	\int_\infty^\infty \exp(-ax^2)dx = \sqrt{{\dfrac{\pi}{a}}}, \quad a > 0;
\end{align}
if we let, \(\dfrac{2k_BT}{m} \equiv a \) we note that this quantity will be positive. We can therefore apply our preceding integral formula to obtain
\begin{align}
	1 = A\sqrt{\dfrac{2\pi k_BT}{m}} \\
	A = \sqrt{\dfrac{m}{2\pi k_BT}}
\end{align}
Which is a constant independent of \(u\) normalizing our one-dimensional Maxwell distribution. 

\section*{Problem 1-3}
\label{sec:1-3}
This problem is on UTIAS computer

\section*{Problem 1-4}
\label{sec:1-4}
Assuming \(p = nk_BT\), and \(T = T_i + T_e\), we obtain
\begin{align}
	p &= nk_B(T_i + T_e) \\
	&= 10^{21}\text{ m}^{-3} \;(40\text{ keV}) \\
	&= 10^{21}\text{ m}^{-3} \;(6.4087\times10^{-15}\text{ J}) \\
	&= 6.41 \text{ MPa} = 64 \text{ atm} = 68 \text{ tons/ft}^2,
\end{align}
for the pressure exerted by a thermonuclear plasma on its container.

\section*{Problem 1-5}
\label{sec:1-5}
We may treat this problem as effectively one-dimensional by assuming that for an infinite plate the state state behaviour of the plasma will depend only on the perpendicular distance from the plate. From here, we state Poisson's equation in one-dimension and assume that the ions have an atomic number of \(Z = 1\).
\begin{align}
	\epsilon_0 \dfrac{d^2 \phi}{dx^2} = -e\left(n_i - n_e \right),
\end{align}
substituting the Boltzmann relation, we obtain
\begin{align}
	\epsilon_0 \dfrac{d^2 \phi}{dx^2} = -e\left[n_0\exp{(q_i\phi/k_BT_i)} - n_0\exp{(q_e\phi/k_BT_e)} \right].
\end{align}
We will assume that \(|\dfrac{e\phi}{k_BT}| \ll 1 \) and we will therefore apply a linear Taylor expansion, and simplify to obtain
\begin{align}
	\epsilon_0 \dfrac{d^2 \phi}{dx^2} \approx n_0e^2\left[\phi/k_BT_i + \phi/k_BT_e \right].
\end{align}
This differential equation yields solutions of the form
\begin{align}
	\phi = \phi_0 \exp{(-|x|/\lambda_D)},
\end{align}
where \(\lambda_D \) is defined as the Debye length, this yields
\begin{align}
	\dfrac{1}{\lambda_D^2} \approx \dfrac{n_0e^2}{\epsilon_0}\left(\dfrac{1}{k_BT_e} + \dfrac{1}{k_BT_i}  \right),
\end{align}
as expected. It can be seen that if \(k_BT_e \gg k_BT_i \) the sum will be dominated by \(k_BT_i \) and we will have
\begin{align}
	\lambda_D = \sqrt{\dfrac{\epsilon_0 k_BT_i}{n_0e^2}}.
\end{align}
The reverse is true for when the electrons are the colder species.

\section*{Problem 1-6}
\label{sec:1-6}






